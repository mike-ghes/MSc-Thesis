\chapter{Conclusions}
\doublespacing

The primary objective of this thesis was to extend \cite{carette2010} by investigating further applications of hybrid sets and functions. 
In this we focused on three main examples: block matrix algebra, integration and convolution of piecewise functions. 
Along the way, several smaller examples were considered as well.
As primarily a demonstration of notation, rational numbers and polynomials were shown as examples of hybrid sets.
Hybrid relations and functions were also demonstrated with maximum flow problems and arithmetic with piece-wise defined functions.


Piece-wise function arithmetic was the first demonstration of the strength of generalized partitions.
Whereas the naive approach to adding an $n$ piece to an $m$ piece function leads to $O(n\cdot m)$ piece function in the 
general case.
Using generalized partitions and careful selection of a generalized partition, we could reduce this to $O(n+m-1)$.
This also required the use of pseudo-functions and leaving functions unevaluated to allow for cancellations to occur first.
If this is not done, we run the risk of attempting to evaluate a function outside of its defined domain.


The same technique can also be applied to adding symbolic block matrices and vectors as was previously shown in
\cite{carette2010}.
Using generalized partitions for multiplication of block matrices is more involved.
By doing so we could formulate an expression using ``wrong ordering'' of breakpoints along the mutual axis of the
multiplicands but still result in the correct product.
Thus we can work with matrices which have symbolically sized blocks without resorting to a case-based approach. 


Hybrid sets were then shown to be a good model for domains of integration.
Numeric integration using hybrid sets as was then shown for both the Riemann and Lebesgue integral.
This allows for more natural manipulation of domains turning $\int_a^b + \int_b^c = \int_a^c$ from a theorem to 
a trivial result of bi-linearity.
As we moved to integration on differential forms, hybrid sets also showed up naturally with the boundary operator.
In conjunction with Stokes' theorem, there are even more opportunities to manipulate  the domains of an integral.


Finally, we applied generalized partitions towards convolution of piecewise functions.
The typical approach for convolving one-piece interval functions involves two cases depending on which interval is longer.
Whether one uses two distinct expressions or commutes the convolution, 
both methods are unsatisfactory when interval bounds
are symbolic and relative length cannot be compared.
With oriented intervals we can use the same equations in a length oblivious manner.
This method handles infinite end points as well without resorting to 6 equations and a 16 case table.


Generalized partitions present a more algebraic way to subdivide an object.
The usefulness of this has long been half-realized for domains of integration 
but there are many areas where they could aptly be applied.
By moving from traditional sets to hybrid sets, we gain a proper notion of a negative set.
Despite the sometimes overlapping notation, this notion should not be confused with set complement
And so the normally imposed condition of disjointness for partitions can be dropped for generalized partitions.
Assuming a good mapping for what is meant by a negative set (generally through the $\op$-reduction $\R[\op]$),
generalized partitions present a useful language.
Even if the examples presented in this thesis are not of direct interest to the reader, hopefully the techniques are
and that they may be applied towards many more endeavors.









