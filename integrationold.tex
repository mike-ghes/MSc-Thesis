\chapter{Integration} \label{integration}
\doublespacing


\section{Lebesgue Integration on $\mathbb{R}^n$}

3-4 pages

\newpage \addtocounter{page}{2} 

\section{Differential Forms}

\begin{definition}
A $k$-form $\beta$ on the open set $\Omega \subset \mathbb{R}^n$ has the form:
\todo[inline]{open set or Lebesgue measurable sets??}
\begin{equation}
\beta = \sum_j b_j(x) \; \text{d} x_{j_1} \wedge ... \wedge \text{d} x_{j_k}
\end{equation}
where $j=(j_1, ..., j_k)$ is a $k$ dimensional multi-index. We say that $\beta \in \Lambda^k(\Omega)$
\end{definition}

We have not yet defined the $\wedge$ operator.

Anti-commutative: $\text{d} x \wedge \text{d} y = - \text{d}y \wedge \text{d} x$. 
Which implies for any permutation $\sigma$ of $\{1,...,k\}$:
\begin{equation}
dx_1 \wedge ... \wedge dx_k = \text{sgn}(\sigma) \; dx_{\sigma(1)} \wedge ... \wedge dx_{\sigma(k)}
\end{equation}

Anti-commutativity additionally implies that for all $x_i$, $dx_i \wedge dx_i = 0$. 

(algebra of differential forms)


\begin{definition}
Let $\alpha$ be a $k$-form on $\Omega \subset \mathbb{R}^n$ of the form $\alpha = A(x) \; \text{d}x_1 \wedge ... \wedge \text{d} x_n$.
If $A \in \mathcal{L}^1 (\Omega , \text{d}x)$ then we define:
\begin{equation}
\int_\Omega \alpha = \int_\Omega A(x) \; \text{d}x
\end{equation}
Where the left-hand side is the integral of a $k$-form and the right-hand side is a Lebesgue integral.
For any $\beta \in \Lambda^k (\Omega)$ we extend this definition linearly as the sum of integrals.
\end{definition}
\todo[inline]{Define $dx$ from $x_1 , ... , x_n$. Need to lift sign change from permutations}


\section{Pull-backs}

Benefit of differential forms is how cleanly they handle changes in coordinates.

\begin{definition}

\end{definition}


\section{Differential Forms}
\begin{definition}
Manifold
\end{definition}

\begin{definition}
Atlas
\end{definition}

\begin{definition}
Differential $k$-form
\end{definition}

Differential form algebra ( $\wedge, \cdot, +$ )

Forms are linear over integrals.

\begin{example}
Simple, concrete example. ``My First Differential Form''
\end{example}

(no hybrid sets in this section) (4 pages)

(Still need to find a spot in this chapter for Lebesgue integrals/Integration of measures)

\newpage \addtocounter{page}{3}

\section{Domains of Integration}

Note on equivalence of $k$-simplices and $k$-rectangles.

\begin{definition}
$k$-rectangle
\end{definition}

\begin{definition}
(singular) $k$-chain as a hybrid set (function) over the $k$-rectangles (with $f$ a differentiable map)
\end{definition}

(2 pages) (+example?)

\newpage \addtocounter{page}{1}

\section{Integrating over $p$-Chains}
\begin{definition}
Integral on a chain
\end{definition}

\begin{example}
Integrate an atlas with overlapping charts (using inclusion-exclusion)
\end{example}

(4 pages)

\newpage \addtocounter{page}{3}

\section{The Boundary Function}

\begin{definition}
$\partial$
\end{definition}

\begin{example}
Boundary of a rectangle
\end{example}

\begin{example}
Boundary of a boundary
\end{example}

\begin{theorem}
$\partial^2 = 0$
\end{theorem}

(3 pages)

\newpage \addtocounter{page}{2}

\section{Stokes' Theorem}
\begin{proof}
(4 pages)
\end{proof}


\newpage \addtocounter{page}{3}

\section{A non-trivial example}
\begin{example}{Tricky Integration}
Using Stokes' theorem and Inclusion/Exclusion to evaluate a tricky theorem like:

\begin{equation}
f(z) = \frac{z^2}{(z^2 + 2z + 2)}
\end{equation}

% http://en.wikipedia.org/wiki/Methods_of_contour_integration

 (2-3 pages)


\end{example}

\newpage \addtocounter{page}{2}

