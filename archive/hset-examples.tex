%%%%%%%%%%%%%%%%%%%%%%%%%%%%%%%%%%%%%%%%%
% Rational Numbers
%%%%%%%%%%%%%%%%%%%%%%%%%%%%%%%%%%%%%%%%%
\subsection{Example: \emph{Rational Arithmetic}}


Any positive rational number can be represented as a hybrid set over the set of primes and vice versa  
(i.e. we have the group isomorphism ().


\begin{example}
	Concretely, we have:
	\begin{equation*}
		20/9 \cdot 15/8 
			= \hset{5^1, 2^2, 3^{-2}} \oplus \hset{5^1, 3^1, 2^{-3}} 
			= \hset{5^2, 2^{-1}, 3^{-1}} 
			= 25/6
	\end{equation*}
\end{example}


Typically one would need to specify equivalence classes on $\mathbb{Q}$ to consolidate the identity ,
but with hybrid set representation, this identity comes for free.
Any common factor between numerator and denominator will cancel result in cancelling multiplicities. 
Equivalence of rational numbers is identical to equivalence on hybrid sets and a normalized hybrid set
For example, $2/4 = \hset{2^1, 2^{-2}}$ which is the un-normalized form of $\hset{2^{-1}} = 1/2$.


One should note that this does not however extend (nicely) for zero and negative $\mathbb{Q}$.
Allowing for our hybrid sets to contain $-1$ would mean we can represent negative rational numbers at the expense of having unique a unique representation. 
The representations of $1/2 = \hset{ 2^{-1}}$ and $-1/-2 = \hset{-1^2, 2^{-1}}$ no longer agree!
Allowing our hybrid sets to contain $0$ leads to rational numbers which are not well defined.
We would like our rational numbers to be members of $\mathbb{Z} \times \mathbb{Z} - \{ 0 \}$ but our construction does not let us discriminate.


%Monic polynomial

%%%%%%%%%%%%%%%%%%%%%%%%%%%%%%%%%%%%%%%%%
% Rational Polynomials
%%%%%%%%%%%%%%%%%%%%%%%%%%%%%%%%%%%%%%%%%
\subsection{Example: \emph{Rational Polynomials}}


Similarly, we can also use hybrid sets to represent monic rational polynomials by encoding the roots and asymptotes.
For example:
\begin{equation}
	\frac{(x-2)}{(x-1)^2(x+1)} = \hset{ 2^1, 1^{-2}, -1^{-1} }
\end{equation}


Although neither model is a revolutionary method of looking at rational numbers and polynomials,
hopefully they demonstrate that hybrid set are not an arcane construction. 
Negative multiplicities of elements can arise very naturally in many contexts.