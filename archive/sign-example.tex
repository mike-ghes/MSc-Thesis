\subsection{Example: \emph{Sign function}}

One would typically write the sign function out as a piecewise function over 3 regions of the extended real line: 
$(-\infty, 0)$, $\{ 0 \}$, and $(0, \infty)$.
Or alternatively using a $+$-reduction using 2 pieces: $(-\infty, 0]$ and $[0, \infty)$.
\begin{equation}
	\text{sign} \;=\; -1^{(-\infty, 0)} \hjoin 0^{\{0\}} \hjoin 1^{(0, \infty)}
	\;=\; \mathcal{R}_+ \left( -1^{(-\infty, 0]} \oplus 1^{[0, \infty)} \right)
\end{equation}
Evaluation of the $\text{sign}$ at the points 1 or 0 is performed as follows:
\begin{equation*}
 \text{sign}(1) = \mathcal{R}_+ \left( -1^{(-\infty, 0]} \oplus 1^{[0, \infty)} \right)(1) 
 = \mathcal{R}_+ \left( -1^{0} \oplus 1^{1} \right)
 = +^0 (-1) +^1 (1) = 1
\end{equation*}
\begin{equation*}
 \text{sign}(0) = \mathcal{R}_+ \left( -1^{(-\infty, 0]} \oplus 1^{[0, \infty)} \right)(0) 
 = \mathcal{R}_+ \left( -1^{1} \oplus 1^{1} \right)
 = +^1 (-1) +^1 (1) = 0
\end{equation*}

Going from 3 regions to 2 may seem a small step.
However, consider two piecewise functions with $n$ and $m$ regions respectively.
Taking the sum of these two functions would lead to a new piecewise function with $n\cdot m$ regions.
We will show that $\op$-reductions can allow us to reduce this to only a linear increase!