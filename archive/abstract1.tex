\addcontentsline{toc}{chapter}{Abstract}
\Large\begin{center}\textbf{Abstract}\end{center}\normalsize
%%  ***  Put your Abstract here.   ***
%% (150 words for M.Sc. and 350 words for Ph.D.)

Mathematic notation has been dominated by sets and, when repeated elements are required, sequences generally make an appearance.
Historical inertia has caused these structures to be used in many situations where they are ill-suited often evidenced by phrases like "without loss of generality", "up to ordering of terms", "up to a sign".
However, by tackling these problems instead with more apt data structures, we can eliminate some of these stipulations and more formally reduce symmetric cases in reasoning.
In particular, this thesis will deal with \emph{hybrid sets} (that is, signed multisets), as well \emph{hybrid functions} (that is, functions with hybrid sets for their domain) with applications in piecewise functions, integration on manifolds, (...).
More than just an aesthetic change, by allowing negative multiplicity (even if it would not make physical sense), we may symblically manipulate structures in ways that might otherwise be cumbersome or inefficient.

\vfill
\textbf{Keywords:} Hybrid set, Signed multiset, Integration on chains, Inclusion-Exclusion, (...)