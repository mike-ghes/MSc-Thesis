\documentclass{letter}
\usepackage{amsmath}
\title{Convolution}
\begin{document}

	I'm investigating a number of uses for \textbf{hybrid sets}: sets with integer ($\pm$) multiplicity rather than boolean
	and hybrid functions: functions with hybrid set domains.
	Union and intersection make sense for boolean sets when the correspond to boolean \texttt{AND} and \texttt{OR}.
	For hybrid sets we use $\oplus$, $\otimes$ and $\ominus$ for pointwise sum, multiplication and subtraction.

	So we can use $[\![ ... )\!)$ for \textbf{oriented intervals} defined as:
	\begin{equation}
		[\![a,b)\!) = [a,b) \ominus [b,a)
	\end{equation}
	Only one of the non-oriented (traditional) intervals  $[a,b) = \{ x | a \leq x < b\}$ or $[b,a) = \{ x | b \leq x < a \}$ 
	will be non-empty.
	The multiplicity of $[\![a,b)\!)$ will be +1 or -1 between $a$ and $b$ depending.
	This gives us some nice properties like:
	\begin{align}
		[\![a,b)\!) &= \ominus [\![b,a)\!) \\
		[\![a,c)\!) &= [\![a,b)\!) \oplus [\![b,c)\!)
	\end{align}
	
	A hybrid function is generally written as $f^A$ where $A$ is a hybrid set.
	Formally it is a hybrid set over ordered pairs that look like $(x,f(x))$ (or sometimes leaving $f$ unevaluated for 
	technical reasons$(x,f)$).
	If $A$ is ``reducible'' (+1 or 0 everywhere) then it behaves like restricting a function to the set $A$.
	A non-reducible set can be reduced through $\mathcal{R}_+$ by adding when multiplicity is positive and subtracting
	when multiplicity is negative (more generally some binary operation and it's inverse).


	The convolution of two functions is defined by:
	\begin{equation}
		(F * G)(t) = \int_{-\infty}^\infty F(\tau) G(t-\tau) d\tau
	\end{equation}
	If $F$ and $G$ are the sum of disjoint sub-functions $f_1 , f_2, \ldots$ and $g_1, g_2, \ldots$ then:
	\begin{equation}
		(F * G)(t) = \sum_i \sum_j \int_{-\infty}^\infty f_i(\tau) g_j(t-\tau) d\tau
	\end{equation}
	So we only need to worry about convolving ``one-piece'' functions then we take the sum of all pairs.
	So for two one piece functions:
	\begin{equation*}
		F(t) = \begin{cases}
		f(t) & t \in [a_f,b_f) \\
		0 & \text{otherwise}
		\end{cases}
		\;\;\;\;\;\;
		G(t) = \begin{cases}
		g(t) & t \in [a_g, b_g) \\
		0 & \text{otherwise}
		\end{cases}
	\end{equation*}
	the typical approach would be to commute $F$ and $G$ so that $b_f - a_f < b_g - a_g$ so that:
	\begin{equation}
	(F * G)(t) = 
		\begin{cases}
			\int_{a_f}^{x-a_g} f(\tau) \; g(t-\tau) \; d\tau 	& (a_f+a_g) \leq t < (b_f+a_g) \\
			\int_{a_f}^{b_f} f(\tau) \; g(t-\tau) \; d\tau		& (b_f+a_g) \leq t < (a_f+b_g) \\
			\int_{x-b_g}^{b_f} f(\tau) \; g(t-\tau) \; d\tau	& (a_f+b_g) \leq t < (b_f+b_g) \\
			0											& \text{otherwise}
		\end{cases}
	\end{equation}
	
	However with hybrid functions we can (regardless of relative length) write:
	\begin{align}
	(F \;*\; G) (t) = 
		\mathcal{R}_+ &\left( \; \left( 
			\int_{a_f}^{t-a_g} f(\tau) \; g(t-\tau) \; d\tau \right)^{[\![a_f+a_g,\; b_f+a_g)\!)} 
				\right. \notag \\ &\oplus \left( 
			\int_{a_f}^{b_f} f(\tau) \; g(t-\tau) \; d\tau \right)^{[\![b_f+a_g,\; a_f+b_g)\!)} 
				\notag \\ &\oplus \left. \left( 
			\int_{t-b_g}^{b_f} f(\tau) \; g(t-\tau) \; d\tau \right)^{[\![a_f+b_g,\; b_f+b_g)\!)} 
				\; \right)(t)
	\end{align}
	
	Let $F, G$ be defined as:
	\begin{align}
		F &= f_1^{(-\infty, -1)} \oplus f_2^{[-1,1)} \oplus f_3^{[1,\infty)} \\
		G &= 0^{(-\infty, -1)} \oplus g_1^{[-1, 0)} \oplus g_2^{[0,1)} \oplus 0^{[1,\infty)}	
	\end{align}	
	then 
	\begin{align}
	(f_2 * g_1)(t) = 
		\mathcal{R}_+ &\left( \; \left( 
			\int_{[\![-1,t+1)\!)} f_2(\tau) \; g_1(t-\tau) \; d\tau \right)^{[\![-2,0)\!)} 
				\right. \notag \\ &\oplus \left( 
			\int_{[\![-1,1)\!)} f_2(\tau) \; g_1(t-\tau) \; d\tau \right)^{[\![0,-1)\!)} 
				\notag \\ &\oplus \left. \left( 
			\int_{[\![t,1)\!)} f_2(\tau) \; g_1(t-\tau) \; d\tau \right)^{[\![-1,1)\!)} 
				\; \right)(t)
	\end{align}
	Suppose we wanted to calculate the exact value at $-.5$. 
	The point is in both $[\![-2,0)\!)$ and $[\![-1, 1)\!)$ once and in $[\![0,-1)\!)$ with multiplicity negative one.
	After $+$-reducing we have:
	\begin{align}
	(f_2 * g_1)(-.5) =& 
		\int_{[\![-1, .5)\!)} f_2(\tau) \; g_1(-.5-\tau) \; d\tau - 
		\int_{[\![-1, 1)\!)} f_2(\tau) \; g_1(-.5-\tau) \; d\tau \notag \\
		& + \int_{[\![-.5, 1)\!)} f_2(\tau) \; g_1(-.5-\tau) \; d\tau \\
		=& \int_{[\![-1, .5)\!) \ominus [\![-1,1)\!) \oplus [\![-.5,1)\!)} f_2(\tau) \; g_1(-.5-\tau) \; d\tau \\
		=& \int_{[\![-.5,.5)\!)} f_2(\tau) \; g_1(-.5-\tau) \; d\tau
	\end{align}
	Note that the terms in (11) are not directly evaluable; $g_1$ may not be well-defined outside of $[-1,0)$.
	However, since the integrands are all identical, we can collect all the terms and the offending domains cancel.
	
	So relative length doesn't matter between $f$ and $g$.
	Now, for what I've been working on this weekend: indeterminate arithmetic (e.g. $+\infty-\infty$) 
	is not a problem either.
	Suppose $a_f = -\infty$ and $b_g = \infty$ are both infinite while $b_f$ and $a_g$ are both finite.
	Which gives the three intervals:
	\begin{equation}
		[\![-\infty, b_f+a_g )\!), [\![ b_f+a_g, \bot )\!), [\![\bot, \infty )\!)
	\end{equation}
	Even though $a_f+b_g$ is unevaluable, the two $\bot$ points are derived identically and so (3) gives us:
	\begin{equation}
		[\![ b_f+a_g, \bot )\!) \oplus [\![\bot, \infty )\!) = [\![b_f+a_g, \infty)\!)
	\end{equation}
	If we assume that $t$ is finite then the domains on the integrals in the 2nd and 3rd terms of (7) are identical:
	\begin{equation}
		[\![a_f, b_f)\!) = [\![-\infty, b_f )\!) = [\![t-b_g, b_f)\!)
	\end{equation}
	These two functions are identical which is important for the identity for hybrid functions: 
	\begin{equation}
		f^A \oplus f^B = f^{A \oplus B}
	\end{equation}
	Finally:
		\begin{align}
	(f^{[-\infty, b_f)} * g^{[a_g, \infty)})(t) = 
		\mathcal{R}_+ &\left( \; \left( 
			\int_{[\![ -\infty, t-a_g)\!)} f(\tau) \; g(t-\tau) \; d\tau \right)^{[\![-\infty,\; b_f+a_g)\!)} 
				\right. \notag \\ &\oplus \left. \left( 
			\int_{[\![ -\infty, b_f )\!)} f(\tau) \; g(t-\tau) \; d\tau \right)^{[\![b_f+a_g,\; \infty)\!)} 
				\; \right)(t)
	\end{align}
	
	Each of the four end-points can either be finite or infinite ($+\infty$ if right end-point, or $-\infty$ if left) so there are
	$2^4=16$ cases.
	
	So far, all cases seem to work, either producing indeterminate end-points which can be removed as above or making 
	empty regions like: $[\![\infty, \infty )\!)$ that can just be ignored. 
	
	
\end{document}