%%%%%%%%%%%%%%%%%%%%%%%%%%%%%%%%%%%%%%%%%
% Inner Product
%%%%%%%%%%%%%%%%%%%%%%%%%%%%%%%%%%%%%%%%%
\subsection{Inner Product}

The inner product or dot product of two vectors is given by:
\begin{equation}
A \cdot B = A^T B = \sum_i A_i B_i
\end{equation}

Returning to the running example of $U^T$ and $V^T$, as defined in (3.13) and (3.14) respectively, we will consider $U^T \cdot V^T$.

\begin{definition}
	Let $X = \hset{x_1^{m_1}, x_2^{m_2}, \ldots , x_n^{m_n} }$ be a hybrid set with elements $x_i$ in a $\mathbb{Z}$-module.
	Given a hybrid function over $X$,  $f^X$, we define the \textbf{sum over $\boldsymbol{f^X}$}, denoted with $\sum$, as
	\begin{equation}
		\sum \! \left( f^X \right)  := \sum_{i=1}^n \left( m_i \cdot f(x_i) \right)
	\end{equation}
	The \textbf{product over $\boldsymbol{f^X}$}, denoted with $\prod$ is defined similarly.
\end{definition}

Then the dot product of $U^T$ and $V^T$ becomes very familiar:
\begin{equation}
	U^T \cdot V^T = \sum \left( (u_i v_i)^{[\![1, k]\!]} 
		\hjoin[\times] (u'_{i-k} v_i)^{(\!(k,\ell]\!]} 
		\hjoin[\times] (u'_{i-k} v'_{i-\ell})^{(\!(\ell,n]\!]} \right)
\end{equation}

The inner expression is identical to $U^T + V^T$ except for a replacement of $+$ with $\times$.




%%%%%%%%%%%%%%%%%%%%%%%%%%%%%%%%%%%%%%%%%
% Outer Product
%%%%%%%%%%%%%%%%%%%%%%%%%%%%%%%%%%%%%%%%%
\subsection{Outer Product}

While the inner product took two $n$-vectors and returned a single number,
the outer product would take those two $n$-vectors and return an $n \times n$ matrix.
Alternatively, one could think of $U^T$ and $V^T$ as $1\times n$ matrices.
Then the inner product is the $1\times 1$ matrix given by $U^T \cdot V$ 
while the outer product is given by $U \cdot V^T$. Formally we define:

\begin{definition}
	Let $A^T = [ a_1, a_2, \ldots, a_n]$ and $B^T = [ b_1, b_2, \ldots, b_m]$,
	then the \textbf{outer product} (or \emph{tensor product}) $\outerproduct$ is given by:
	\begin{equation}
		A \outerproduct B = A \; B^T = \left[
			\begin{array}{ccc}
				a_1 b_1 & \ldots 	& a_1 b_m \\
				\vdots 	& \ddots & \vdots \\
				a_n b_1 & \ldots 	& a_n b_m
			\end{array}
		\right]
	\end{equation}
\end{definition}

Hybrid sets won't allow us to do anything differently.

But it will provide a nice lead in to our notation for matrix multiplication.