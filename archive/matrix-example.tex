\documentclass{letter}
\usepackage{amsmath}
\title{Abstract Matrix Multiplication}
\begin{document}

Assume two matrices which we plan to multiply $A$ which is $n \times m$ and $B$ which is $m \times p$.
Both are block matrices:
\begin{equation}
	A = \left[ \begin{array}{cc} A_{11} & A_{12} \\ A_{21} & A_{22} \end{array} \right]
	\;\;\;\;\; \text{and} \;\;\;\;\;
	B = \left[ \begin{array}{cc} B_{11} & B_{12} \\ B_{21} & B_{22} \end{array} \right]
\end{equation}
Where $A_{11}$ is a $k_1 \times k_2$ matrix and $B_{11}$ is a $\ell_1 \times \ell_2$ matrix.
Note that $0 \leq k_2 , \ell_1 \leq m$ but the ordering of $k_2$ and $\ell_1$ is unknown.
Multiplication with $k_2 = \ell_1$,  is commonly used:
\begin{equation}
	AB = \left[ \begin{array}{cc} 
		A_{11}B_{11}+A_{12}B_{21} & A_{11}B_{12}+A_{12}B_{22} \\ 
		A_{21}B_{11}+A_{22}B_{21} & A_{21}B_{12}+A_{22}B_{22}
	\end{array} \right]
\end{equation}
(``conformable partitioning'' is the term used in \emph{Elementary Matrix Theory} - Howard Eves).
I haven't found anything written on non-conformable partitioned multiplication but admittedly haven't looked very hard.


For the general case ($k_2 \neq \ell_1$), we still get back a $2 \times 2$ block matrix, which we will denote $C$.
\begin{equation}
	AB = \left[ \begin{array}{cc} C_{11} & C_{12} \\ C_{21} & C_{22} \end{array} \right]
\end{equation}
$C_{11}$ is a $k_1 \times \ell_2$ sub-matrix; the sizes of the other partitions can be derived from this.
The domains of these 12 submatrices can be partitioned using:
\begin{equation*}\begin{array}{cc}
	N_1 = [\![0,k_1)\!) & N_2 = [\![k_1, n)\!) 
\end{array}\end{equation*}
\begin{equation*}\begin{array}{cc}
	P_1 = [\![0, \ell_2)\!) & P_2 = [\![ \ell_2, p)\!)
\end{array}\end{equation*}
\begin{equation*}\begin{array}{ccc}
	M_1 = [\![0,k_2)\!) & M_2 = [\![ k_2, \ell_1)\!) & M_3 = [\![ \ell_1, m)\!)
\end{array}\end{equation*}
So we can rewrite $A$ and $B$ as:
\begin{equation}
	A = 	A_{11}^{N_1 \times M_1} \oplus A_{12}^{N_1 \times (M_2 \oplus M_3)} \oplus 
			A_{21}^{N_2 \times M_1} \oplus A_{22}^{N_2 \times (M_2 \oplus M_3)}
\end{equation}
\begin{equation}
	B = 		B_{11}^{(M_1 \oplus M_2) \times P_1} \oplus B_{12}^{(M_1 \oplus M_2) \times P_2} \oplus 
			B_{21}^{M_3 \times P_1} \oplus B_{12}^{M_3 \times P_2}
\end{equation}

Here $\oplus$ is \emph{not} the direct sum of matrices, but hybrid function pointwise sum. 
$\times$ is the Cartesian product of intervals. 

For $i,j \in \{ 1,2 \}$ we can write blocks in $C$ as:
\begin{equation}
	C_{i,j} 	= A_{i,1}^{N_i \times M_1} B_{1,j}^{M_1 \times P_j} 
			+ A_{i,2}^{N_i \times M_2} B_{1,j}^{M_2 \times P_j}
			+ A_{i,2}^{N_i \times M_3} B_{2,j}^{M_3 \times P_j}
\end{equation}

If $k_2 = \ell_1$ then $M_2 = \emptyset$.
We can think of multiplying a $n \times 0$ matrix by a $0 \times p$ matrix as giving a $n \times p$ matrix which is the sum
over an empty set, 0 everywhere.

If $k_2 < \ell_1$ then this is like treating $A$ and $B$ instead as $2 \times 3$ and $3 \times 2$ block matrices.
$M_1$, $M_2$ and $M_3$ are all positive intervals and form conformable blocks.

If $k_2 > \ell_1$ then $M_2$ is a negative interval which doesn't have a good interpretation.
To simplify $C_i,j$, we can use the partition $ \{ M_1 \oplus M_2, \ominus M_2, M_2 \oplus M_2 \}$ which contains only
positive intervals.


First we rewrite the hybrid functions in the first and third terms:


\begin{align}
	C_{i,j} 	= &\left( A_{i,1}^{N_i \times M_1 \oplus M_2} \oplus A_{i,1}^{N_i \times \ominus M_2} \right)
				\left( B_{1,j}^{M_1 \oplus M_2 \times P_j} \oplus B_{1,j}^{\ominus M_2 \times P_j} \right)
			+ A_{i,2}^{N_i \times M_2} B_{1,j}^{M_2 \times P_j} \notag\\
			&+ \left( A_{i,2}^{N_i \times \ominus M_2} \oplus A_{i,2}^{N_i \times M_3 \oplus M_2} \right)
			 \left( B_{2,j}^{\ominus M_2 \times P_j} \oplus B_{2,j}^{M_3 \oplus M_2 \times P_j} \right)
\end{align}

Which are now the product of conformable $1 \times 2$ and $2 \times 1$ block matrices.
This gives us a $1 \times 1$ block matrix :
\begin{align}
	C_{i,j} 	= & A_{i,1}^{N_i \times M_1 \oplus M_2} B_{1,j}^{M_1 \oplus M_2 \times P_j}
			+ A_{i,1}^{N_i \times \ominus M_2} B_{1,j}^{\ominus M_2 \times P_j}
			+ A_{i,2}^{N_i \times M_2} B_{1,j}^{M_2 \times P_j} \notag\\
			&+ A_{i,2}^{N_i \times \ominus M_2} B_{2,j}^{\ominus M_2 \times P_j}
			+ A_{i,2}^{N_i \times M_3 \oplus M_2} B_{2,j}^{M_3 \oplus M_2 \times P_j}
\end{align}

Then we merge the middle three terms:
\begin{align}
	C_{i,j} 	=& \left( A_{i,1}^{N_i \times \ominus M_2} 
				\oplus  A_{i,2}^{N_i \times M_2} 
				\oplus A_{i,2}^{N_i \times \ominus M_2} \right)
			\left( B_{1,j}^{\ominus M_2 \times P_j} 
				\oplus B_{1,j}^{M_2 \times P_j} 
				\oplus B_{2,j}^{\ominus M_2 \times P_j} \right)\notag\\
			&+ A_{i,1}^{N_i \times M_1 \oplus M_2} B_{1,j}^{M_1 \oplus M_2 \times P_j} 
			+ A_{i,2}^{N_i \times M_3 \oplus M_2} B_{2,j}^{M_3 \oplus M_2 \times P_j} 
\end{align}


$A_{i,2}$ occurs twice with opposite sign on the same region and so cancels itself. As does $B_{1,j}$:
\begin{align}
	C_{i,j} 	=& \left( A_{i,1}^{N_i \times \ominus M_2} \oplus  A_{i,2}^{N_i \times \emptyset}  \right)
			\left(  B_{2,j}^{\ominus M_2 \times P_j} \oplus B_{1,j}^{\emptyset \times P_j} \right)\notag\\
			&+ A_{i,1}^{N_i \times M_1 \oplus M_2} B_{1,j}^{M_1 \oplus M_2 \times P_j} 
			+ A_{i,2}^{N_i \times M_3 \oplus M_2} B_{2,j}^{M_3 \oplus M_2 \times P_j} 
\end{align}
And functions over empty domains can be removed:

\begin{align}
			= & A_{i,1}^{N_i \times M_1 \oplus M_2} B_{1,j}^{M_1 \oplus M_2 \times P_j} 
			+ A_{i,1}^{N_i \times \ominus M_2} B_{2,j}^{\ominus M_2 \times P_j}
			+ A_{i,2}^{N_i \times M_3 \oplus M_2} B_{2,j}^{M_3 \oplus M_2 \times P_j}
\end{align}


Formally this relies on the following identities:

\begin{equation}
	M^B \oplus M^C = M^{B\oplus C}
\end{equation}
(we can split up a matrix into blocks)

\begin{equation}
	M_1^A \oplus M_2^\emptyset = M_1^A = M_2^\emptyset \oplus M_1^A
\end{equation}
(empty blocks can be ignored)

\begin{equation}
	M_1^{A \times B} M_2^{B \times D} + M_3^{A \times C} M_4^{C \times D} 
	= \left(M_1^{A\times B} \oplus M_3^{A \times C} \right) \left( M_2^{B \times D} \oplus M_4^{C \times D} \right)
\end{equation}




\end{document}