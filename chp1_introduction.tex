\chapter{Introduction}
\doublespacing


%%%%%%%%%%%%%%%%%%%%%%%%%%%%%%%%%%%%%%%%%
% MOTIVATION
%%%%%%%%%%%%%%%%%%%%%%%%%%%%%%%%%%%%%%%%%
\section{Motivation}


%(Unordered pairs \cite{EWD:EWD1223} ?)	
Some data structures in mathematics are more widely adopted than others and none more than Cantor's notion of sets.
The very foundations of mathematics lie in set theory: numbers are defined in terms of sets as are ordered tuples 
which in turn lead to relations, functions, sequences and from there branches into countless other structures.
But this trunk typically omits a satisfactory treatment of several \emph{generalized sets}.
For example, it is difficult to begin speaking about \emph{hybrid sets} without immediately punctuating, 
\emph{``---that is, multi-sets with negative multiplicity''}.


It is a statement of progress that multi-sets have even entered into common mathematical parlance.
On the other hand, sequences need no introduction and one can easily represent a multi-set by a sequence: 
if an element occurs in a multi-set with multiplicity $n$, ensure that it's in the corresponding sequence $n$ times as well.
In this way, sequences are often used \emph{in place of} multi-sets with the added structure of an ordering.
Indeed,  \emph{``why not?''}, even if the order of elements is not needed, \emph{``surely it can't hurt?''}


Consider the \emph{Fundamental Theorem of Arithmetic}:
\begin{quote}
	``Every positive integer, except 1, is a product of primes. 
	(\ldots) 
	The standard form of $n$ is unique; 
	apart from rearrangement of factors, $n$ can be expressed as a product of primes in one way only.''
	\attrib{Hardy and Wright 1979, p.2-3}
\end{quote}
By recognizing the possibility of rearranging factors, the authors implicitly define type the ``product of primes'' as a sequence rather than, more aptly a multi-set.
For iterated, commutative operators (e.g. $\sum, \prod, \bigcap, \bigcup$), the order of terms is irrelevant so then, 
\emph{``why order terms to begin with?''}
Reisig \cite{reisig1985petri} uses multi-sets to define relation nets where 
``\ldots several individuals of some sort do not have to be distinguished", and moreover
 ``One \emph{should not} be forced to distinguish individuals if one doesn't wish to.  This would lead to overspecification''
(emphasis added). The same applies here.


Secondly --- and this is a more subtle and subjective point --- 
the empty sequence is not nearly as enshrined as the empty set.
Whereas the empty set is often the first thing that comes to mind when searching for trivial cases or counter-examples,
the empty sequence is not usually so readily recalled.
Despite the empty sequence being just as well-defined as the empty set; it tends to be treated as an aberrant case.
In the case of iterated operators over an empty set, it is simply the respective identity.
In the case of $\prod$, this is 1, and so 1 \emph{is} a product of primes.
It is uniquely, the product of the empty set of primes.
Similarly, a sequence containing just one element is still a sequence but this also can be easy to forget.
Some definitions will also disregard this to say, ``is prime or the product of primes''. 
Whether this is extra specificity is a fault to the authors or simply a reminder to readers, the point remains the same.
The perception of sequences can be misleading in ways in which sets are not often misconstrued.
Stripped of these qualifications we are left with the more succinct:
\begin{quote}
	``Every positive integer is the product of a unique multiset of primes.''
\end{quote}


It goes deeper than a matter of aesthetics. 
Sets are certainly flexible objects and it is tempting to take a minimal approach to the tools required in one's toolbox.
But the algebra of sets, is a fairly restrictive one.
Consider the sets $[a,b) = \{ x \; |\; a \leq x < b \}$ and $[b,c) = \{ x \;|\: b \leq x < c \}$.
Their union $[a,b) \cup [b,c)$ will depend heavily on the relative ordering of $a$, $b$ and $c$. 
More often, as in integration what one really wants is $[a,b) + [b,c) = [a,c)$.
When boolean algebra and numeric algebra interface, results can get unnecessarily messy.
Hybrid sets will allow us to remove this tension, by committing fully to numeric algebra
In this spirit that we will graft hybrid sets into areas of mathematics where current approaches are unsatisfactory.





%%%%%%%%%%%%%%%%%%%%%%%%%%%%%%%%%%%%%%%%%
% OBJECTIVES
%%%%%%%%%%%%%%%%%%%%%%%%%%%%%%%%%%%%%%%%%		
\section{Objectives}


This thesis will include and extend the work of \cite{carette2010} on hybrid sets and their applications.
In particular we will take as inspiration the following identity for integrals:
\begin{equation}
	\int_a^b f(x) \diff x = -\int_b^a f(x) \diff x
\end{equation}
This allows for the identity
\begin{equation}
	\int_a^c f(x) \diff x = \int_a^b f(x) \diff x + \int_b^c f(x) \diff x
\end{equation}
regardless of the ordering of $a$, $b$ and $c$.
One can think of the region from $a$ to $c$ being partitioned into $a$ to $b$ and $b$ to $c$.
But this is not a partition in the usual sense.
Rather than the union of disjoint subsets, it is a sum of oriented subsets: a generalized partition.
This is a rather nice behaviour that would be nice to emulate in other contexts as well.


We will consider several different applications.
When matrices are represented in terms of block submatrices, 
the sum or product can vary depending on operand block sizes.
Representing the regions of a block matrix with hybrid sets will allow for all cases to be consolidated into one;
no matter how those blocks may overlap.
Lebesgue integration relies on measurable sets and also does not naturally support orientation.
Here, hybrid sets will allow us to evaluate integrals like $\int_1^0 1_{\mathbb{R} \setminus \mathbb{Q}} \diff x = -1$
Finally, we will consider the convolution of piecewise interval functions.
Typically different equations are used depending on the relative length of intervals but with hybrid sets we can condense this
into one general equation and ignore the interval lengths altogether.


Unifying all of this is the argument that hybrid sets are a very useful structure and there are many instances where existing
structures ought to be replaced.
Hopefully these examples will provide inspiration to the reader to recognize similar situations elsewhere as well.




%%%%%%%%%%%%%%%%%%%%%%%%%%%%%%%%%%%%%%%%%
% RELATED WORK
%%%%%%%%%%%%%%%%%%%%%%%%%%%%%%%%%%%%%%%%%
\section{Related Work}


It is difficult to date the origin of multisets.
Although the term itself was coined by De Bruijn in corresponces with Knuth \cite{knuth2014art},
the thought of a ``collection of objects that may or may not be distinguished'' is as old as tally marks. 
In regards to the generalization to \emph{signed} multisets, Hailperin \cite{hailperin1986boole} 
suggests that Boole's 1854 \emph{Laws of Thought} \cite{boole1854investigation} is actually a treatise of signed multisets.
Whether this was Boole's intent is debateable.
Sets with negative membership explicitly began to appear in \cite{whitney1933characteristic} and were 
formalized under the name Hybrid sets in Blizard's extensive work with generalized sets \cite{blizard1988, blizard1990} 
Although hybrid set and signed multiset are the most common nomenclature, other names appearing in literature include
multiset (specifying positive when speaking of unsigned multisets) \cite{reisig1985petri} 
and integral multiset \cite{wildberger2003new}. 


Existing explicit applications of hybrid sets are currently limited.
Loeb \emph{et al.} \cite{damiani1991, loeb1992} use hybrid sets to generalize several combinatoric identities to negative values.
Bailey \emph{et al.} \cite{bailey2009hypergraphic} and Ban\^{a}tre \emph{et al.} \cite{banatre2006} 
have also had success with hybrid sets in chemical programming. 
Representing a solution is represented as a collection of atoms and molecules, 
negative multiplicities are treated as ``antimatter'' . 
For a deeper overview and systemization of generalized sets, see \cite{singh2007, singh2008systematization}.
These these ideas allow symbolic computation on functions defined piecewise \cite{carette2010}.


%%%%%%%%%%%%%%%%%%%%%%%%%%%%%%%%%%%%%%%%%
% MOTIVATION
%%%%%%%%%%%%%%%%%%%%%%%%%%%%%%%%%%%%%%%%%
\section{Thesis Outline}


In chapter 2, the foundations for hybrid sets and functions with hybrid set domains will be laid. 
This will provide us with formal definitions and some immediate applications to piecewise functions will be presented.
Chapter 3 will see hybrid domains applied towards symbolic matrix algebra.
Addition has already been considered \cite{carette2010}, but this will be extended to multiplication as well.
In chapter 4, hybrid functions will be applied towards integration. 
We will start from foundations and use hybrid functions to allow for an oriented Lebesgue integral.
We will then show how hybrid sets naturally come up in Stokes' theorem with the boundary operator.
Finally in chapter 5, hybrid sets will be applied towards convolution of piecewise functions.


















