\chapter{Symbolic Linear Algebra}

In mathematics literature, is common practice to use ellipses: ``$\ldots$'' in matrices.
For example:
\begin{equation}
	\left[
		\begin{array}{cccc}
			A_{1,1} & A_{1,2}	& \ldots 	& A_{1,n} \\
			0		& A_{2,2}	& \ldots	& A_{2,n} \\
 			\vdots 	& \ddots 	& \ddots & \vdots \\
			0		& \ldots 		& 0 		& A_{n,n} \\
		\end{array}
	\right]
\end{equation}
and symbolic blocks as in
\begin{equation}
	\left[
		\begin{array}{c|c}
			A & B \\
			\hline
			C & D \\
		\end{array}
	\right]
\end{equation}
where $A,B,C,D$ are sub matrices.

Computer algebra has unsatisfactory structures to represent these  

\todo[inline]{more}


\newpage

%%%%%%%%%%%%%%%%%%%%%%%%%%%%%%%%%%%%%%%%%
%
% INTERVALS
%
%%%%%%%%%%%%%%%%%%%%%%%%%%%%%%%%%%%%%%%%%

\section{Oriented Intervals}

\begin{definition}
	Given a totally ordered set $(X, \leq)$ \emph{(and with an implied strict ordering $<$)}, 
	for any $a,b \in X$, an \textbf{interval between $\boldsymbol{a}$ and $\boldsymbol{b}$} 
	is the set of elements in $X$ between $a$ and $b$, up to inclusion of $a$ and $b$ themselves. 
	Formally:
	\begin{align}
		[a,b] = \{ x \in X \;|\; a \leq x \leq b \} \\
		[a,b) = \{ x \in X \;|\; a \leq x < b \} \\
		(a,b] = \{ x \in X \;|\; a < x \leq b \} \\
		(a,b) = \{ x \in X \;|\; a < x < b \}
	\end{align}
\end{definition}

It should be noted that for $b<a$, $[b,a]$ is the empty set. 
Also, the interval $[a,a]$ contains a single point while $(a,a)$, $(a,a]$, and $[a,a)$ are all empty.
As intervals are simply sets, they can naturally be interpreted as hybrid sets.
If $a \leq b \leq c$, for intervals $[a,b)$ and $[b,c)$ using the hybrid set operator $\oplus$, one has
$[a,b) \oplus [b,c) = [a,c)$
In this case, $\oplus$ behaves like concatenation but this is not always true.
When $a \leq c \leq b$ then $[a,b) \oplus [b,c) = [a,b)$.
When working with intervals, a case-based approach to consider relative ordering of endpoints easily becomes quite cumbersome.
Thus we turn to oriented intervals.
\todo[inline]{1 more line}

\begin{definition}
	We define \textbf{oriented intervals} with $a,b\in X$, where $X$ is a totally ordered set, 
	using hybrid set point-wise subtraction as follows:
	\begin{align}
		[\![ a,b )\!) = [a,b) \ominus [b,a) \\
		(\!( a,b ]\!] = (a,b] \ominus (b,a] \\
		[\![ a,b ]\!] = [a,b] \ominus (b,a) \\
		(\!( a,b )\!) = (a,b) \ominus [b,a]
	\end{align}
\end{definition}

For any choice of distinct $a$ and $b$, exactly one term will be empty.
Unlike traditional interviews where $[a,b)$ would be empty if $b < a$,  
the oriented interval $[\![a,b)\!)$ simply has elements with negative multiplicity.
Several results follow immediately from this definition.

\begin{theorem} For all $a,b,c \in \mathbb{R}$, 
	\begin{align}
		[\![a,b)\!) &= \ominus [\![b,a)\!) \\
		(\!(a,b]\!] &= \ominus (\!(b,a]\!] \\
		[\![a,b]\!] &= \ominus (\!(a,b)\!) \\
		(\!(a,b)\!) &= \ominus [\![a,b]\!]
	\end{align}
\end{theorem}

We should make a note here how oriented intervals behave when $a=b$.
Like their unoriented analogues, the oriented intervals $[\![ a,a )\!)$ and $(\!( a,a ]\!]$ are both empty 
while $[\![a,a]\!]$ contains the point at $a$ (with multiplicity 1).
However, unlike traditional intervals $(\!(a,a)\!)$ is \emph{not} empty but rather, $(\!(a,a)\!) = \ominus [\![a,a]\!] = \hset{a^{-1}}$.

\todo[inline]{more}

\begin{theorem}
	For all $a,b,c \in \mathbb{R}$ (regardless of relative ordering),
	\begin{equation}
		[\![ a,b )\!) \oplus [\![ b,c )\!) = [\![ a,c )\!)
	\end{equation}
\end{theorem}

\begin{proof}
	$[\![a,b)\!) \oplus [\![ b,c )\!)$ 

	$= \left( [a,b) \ominus [b,a) \right) \oplus \left( [b,c) \ominus [c,b) \right)$ 

	$= \left( [a,b) \oplus [b,c) \right) \ominus \left( [c,b) \oplus [b,a) \right)$

	If $a \geq c$ then $[c,a) = \emptyset$ and so $[\![a,c)\!) = [a,c)$. 
	\begin{description}
		\item[Case 1: $a \leq b \leq c$] then $[c,b) = [b,a) = \emptyset$ and $[a,b) \oplus [b,c) = [a,c)$
		\item[Case 2: $b \leq a \leq c$] then $[b,c) \ominus [b,a) = [b,a) \oplus [a,c) \ominus [b,a) = [a,c)$
		\item[Case 3: $a \leq c \leq b$] then $[a,b) \ominus [c,b) = ([a,c) \oplus [c,b)) \ominus [c,b) = [a,c)$
	\end{description}
	Similar arguments will show that when $c \geq a$, that $[\![a,b)\!) \oplus [\![ b,c )\!) = \ominus [a,c)$.
	
\end{proof}

This sort of reasoning is routine but a constant annoyance when dealing with intervals and is exactly the reason we want to be working with oriented intervals.
Many similar formulations such as $[\![ a,b ]\!] \oplus (\!( b,c )\!) = [\![a,c)\!)$ are also valid for any ordering of $a,b,c$.
We will not enumerate all possible cases here.

\todo[inline]{Note about partitions}

\newpage




%%%%%%%%%%%%%%%%%%%%%%%%%%%%%%%%%%%%%%%%%
%
% VECTORS
%
%%%%%%%%%%%%%%%%%%%%%%%%%%%%%%%%%%%%%%%%%

\section{Symbolic Vectors}

\todo[inline]{Vectors as hybrid functions}

We will use the following $n$-dimensional vectors as a running example in this section:

\begin{align}
	U^T &= [ u_1, u_2, \ldots, u_{k}, u'_1, u'_2, \ldots, u_{n-k} ] \\
	V^T &= [ v_1, v_2, \ldots, v_{\ell}, v'_1, v'_2, \ldots, v_{n-\ell} ]
\end{align}

Using intervals, these vectors can be represented by hybrid functions over their indices.

For example
\begin{align}
	U^T &= (i \mapsto u_i)^{[\![1, k]\!]} \oplus (i \mapsto u_{i-k})^{(\!(k,n]\!]} \\
	V^T &= (i \mapsto v_i)^{[\![1, \ell]\!]} \oplus (i \mapsto v_{i-\ell})^{(\!(\ell,n]\!]}
\end{align}
Although for clarity and succinctness we will use $(u_i)$ instead of $(i \mapsto u_i)$.

However we are more interested in performing arithmetic with these vectors.




%%%%%%%%%%%%%%%%%%%%%%%%%%%%%%%%%%%%%%%%%
% Addition
%%%%%%%%%%%%%%%%%%%%%%%%%%%%%%%%%%%%%%%%%
\subsection{Vector Addition}

Consider pointwise vector addition $U^T + V^T$:
\begin{align}
	U^T + V^T 
	&= \left( (u_i)^{[\![1, k]\!]} \oplus (u'_{i-k})^{(\!(k,n]\!]} \right) 
		\hjoin[+] 
		\left( (v_i)^{[\![1, \ell]\!]} \oplus (v'_{i-\ell})^{(\!(\ell,n]\!]} \right) \\
	&= \left( (u_i)^{[\![1, k]\!]} \oplus (u'_{i-k})^{(\!(k,\ell]\!]} \oplus (u'_{i-k})^{(\!(\ell,n]\!]} \right) 
		\hjoin[+]
		\left( (v_i)^{[\![1, k]\!]} \oplus (v_i)^{(\!(k, \ell]\!]} \oplus (v'_{i-\ell})^{(\!(\ell,n]\!]} \right) \\
	&= \left( (u_i + v_i)^{[\![1, k]\!]} 
		\hjoin[+] (u'_{i-k} + v_i)^{(\!(k,\ell]\!]} 
		\hjoin[+] (u'_{i-k}+v'_{i-\ell})^{(\!(\ell,n]\!]} \right)
\end{align}

This formulation is not unique.

The choice to partition $[\![1,n]\!]$ into $[\![1,k]\!] \oplus (\!(k,\ell]\!] \oplus (\!(\ell, n]\!]$ was arbitrary.

We can just as easily partition $[\![1,n]\!]$ into $[\![1,\ell]\!] \oplus (\!(\ell, k]\!] \oplus (\!(k, n]\!]$ to get the equivalent expression:

\begin{equation}
	U^T + V^T = \left( (u_i + v_i)^{[\![1, \ell]\!]} 
		\hjoin[+] (u_{i} + v'_{i-\ell})^{(\!(\ell,k]\!]} 
		\hjoin[+] (u'_{i-k}+v'_{i-\ell})^{(\!(k,n]\!]} \right)
\end{equation}

We must be careful while evaluating these expressions to not forget that $(u'_{i-k} + v_i)$ is actually shorthand for the function:
\begin{equation*}
	i \mapsto u'_{i-k} + v_i
\end{equation*}

For example, consider the concrete example where $n=5$, $k=4$ and $\ell = 1$ so that
$U^T = [ u_1, u_2, u_3, u_4, u'_1 ]$ and
$V^T = [ v_1, v'_1, v'_2, v'_3, v'_4 ]$.

We will also only assume that the functions $u_i, u'_i, v_i$ and $v'_i$ are defined only on the intervals in which they appear (e.g. $u_5$ is undefined, as is $v'_1$).

Then the expression in (3.19) becomes:
\begin{equation}
(u_i + v_i)^{[\![1,4]\!]} \hjoin[+] (u'_{i-4} + v_i)^{(\!(4,1]\!]} \hjoin[+] (u'_{i-4} + v'_{i-1})^{(\!(1,5]\!]}
\end{equation}

None of the individual subterms cannot be evaluated directly.

In the first term, $v_i$ is not totally defined over the interval $[\![1,4]\!]$.

In the third term, on the interval $(\!(1,5]\!]$, $u'_{i-4}$ would even evaluated on negative indices.

However, these unevaluable terms also appear in the middle term however the interval $(\!(4,1]\!]$ is a negatively oriented interval and the offending points cancel!




%%%%%%%%%%%%%%%%%%%%%%%%%%%%%%%%%%%%%%%%%
% Inner Product
%%%%%%%%%%%%%%%%%%%%%%%%%%%%%%%%%%%%%%%%%
\subsection{Inner Product}

The inner product or dot product of two vectors is given by:
\begin{equation}
A \cdot B = \sum_i A_i B_i
\end{equation}

Returning to the running example of $U^T$ and $V^T$, as defined in (3.13) and (3.14) respectively, we will consider $U^T \cdot V^T$.

\begin{definition}
	Let $X = \hset{x_1^{m_1}, x_2^{m_2}, \ldots , x_n^{m_n} }$ be a hybrid set with elements $x_i$ in a $\mathbb{Z}$-module.
	Given a hybrid function over $X$,  $f^X$, we define the \textbf{sum over $\boldsymbol{f^X}$}, denoted with $\sum$, as
	\begin{equation}
		\sum \! \left( f^X \right)  := \sum_{i=1}^n \left( m_i \cdot f(x_i) \right)
	\end{equation}
	The \textbf{product over $\boldsymbol{f^X}$}, denoted with $\prod$ is defined similarly.
\end{definition}

Then the dot product of $U^T$ and $V^T$ becomes very familiar:
\begin{equation}
	U^T \cdot V^T = \sum \left( (u_i v_i)^{[\![1, k]\!]} 
		\hjoin[\times] (u'_{i-k} v_i)^{(\!(k,\ell]\!]} 
		\hjoin[\times] (u'_{i-k} v'_{i-\ell})^{(\!(\ell,n]\!]} \right)
\end{equation}

The inner expression is identical to $U^T + V^T$ except for a replacement of $+$ with $\times$.




%%%%%%%%%%%%%%%%%%%%%%%%%%%%%%%%%%%%%%%%%
% Outer Product
%%%%%%%%%%%%%%%%%%%%%%%%%%%%%%%%%%%%%%%%%
\subsection{Outer Product}

While the inner product took two $n$-vectors and returned a single number,
the outer product would take those two $n$-vectors and return an $n \times n$ matrix.
Alternatively, one could think of $U^T$ and $V^T$ as $1\times n$ matrices.
Then the inner product is the $1\times 1$ matrix given by $U^T \cdot V$ 
while the outer product is given by $U \cdot V^T$. Formally we define:

\begin{definition}
	Let $A^T = [ a_1, a_2, \ldots, a_n]$ and $B^T = [ b_1, b_2, \ldots, b_m]$,
	then the \textbf{outer product} (or \emph{tensor product}) $\otimes$ is given by:
	\begin{equation}
		A \boxtimes B = A \; B^T = \left[
			\begin{array}{ccc}
				a_1 b_1 & \ldots 	& a_1 b_m \\
				\vdots 	& \ddots & \vdots \\
				a_n b_1 & \ldots 	& a_n b_m
			\end{array}
		\right]
	\end{equation}
\end{definition}

Hybrid sets won't allow us to do anything differently.

But it will provide a nice lead in to our notation for matrix multiplication.




%%%%%%%%%%%%%%%%%%%%%%%%%%%%%%%%%%%%%%%%%
%
% MATRICES
%
%%%%%%%%%%%%%%%%%%%%%%%%%%%%%%%%%%%%%%%%%
\section{Abstract Matrices}

It is common practice in mathematics to represent matrices symbolically with sub-matrices such as:
\begin{equation}
	A = \left[
		\begin{array}{ccc}
			A_1 & A_2 \\
			A_3 & A_4 \\
		\end{array}
	\right]
\end{equation}
If $A$ is an $n \times n$ matrix then $A_1, A_2, A_3, A_4$ are not entries but $(k \times \ell)$, $(n-k \times \ell)$, $(k \times n - \ell)$ and $(n-k \times n - \ell)$ matrices respectively.
Ellipses are also routinely used for interpolating over regions of a matrix, as in:
\begin{equation}
	M = \left[
		\begin{array}{ccc}
			x_{11} & \ldots & x_{1n} \\
			& \ddots & \vdots \\
			0 & & x_{nn}
		\end{array}
	\right]
\end{equation}

%%%%%%%%%%%%%%%%%%%%%%%%%%%%%%%%%%%%%%%%%
% Matrix Addition
%%%%%%%%%%%%%%%%%%%%%%%%%%%%%%%%%%%%%%%%%
\subsection{Matrix Addition}


%%%%%%%%%%%%%%%%%%%%%%%%%%%%%%%%%%%%%%%%%
% Matrix Multiplication
%%%%%%%%%%%%%%%%%%%%%%%%%%%%%%%%%%%%%%%%%
\subsection{Matrix Multiplication}


























