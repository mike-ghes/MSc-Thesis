\chapter{Generalized Partitions}
\section{Piecewise Functions}
\doublespacing


\todo[inline]{To Do: Blending ``piecewise are everywhere''}%%%%%%%%%%%%%%%%

The perennial example of a piecewise function is $\mathrm{abs}:\mathbb{R} \to \mathbb{R}_{+} \cup \{ 0 \}$ given in the form:

\begin{equation}
\mathrm{abs}(x) = 
  \left\{
     \begin{array}{lr}
       -x & : x < 0 \\
       x & : x \geq 0
     \end{array}
   \right.
\end{equation}

To evaluate abs for an argument $x$, one must first determine which sub-function to use. 
If $x < 0$ then the first case is evaluated and abs will return the result of $x \mapsto -x$. 
Otherwise, if $x \geq 0$ the second case is evaluated and the result of $x \mapsto x$ is returned. 
Rather than as a condition, we could just as easily think of ``$x<0$'' and ``$x \geq 0$'' as partitions of the real line.
Evaluation then occurs by checking whether $x\in \mathbb{R}_{+} \cup \{ 0 \} $ or $x \in \mathbb{R}_{-}$.
In general, a piecewise function $f$ will take the form:

\begin{equation}
f(x) = 
  \left\{
     \begin{array}{lr}
       f_1(x) & : x \in P_1 \\
       f_2(x) & : x \in P_2 \\ 
       \vdots & \vdots \\
       f_n(x) & : x \in P_n
     \end{array}
   \right.
   \label{eq_fP}
\end{equation}
where the set $\{ P _ i \}$ forms a partition of the domain of $f$ and for each $f_i$ is defined over all of the corresponding $P_i$. To formalize this we require the ability to restrict a function's domain and join disjoint pieces together.

\begin{definition}
Given a function $f:X \to Y$ for any subset of the domain, $Z \subset X$, the \emph{restriction of $f$ to $Z$} is the function $f|_Z : Z \to Y$, such that $f|_Z(x) = f(x)$ for all $x \in Z$.
\end{definition}



\begin{definition}
Define $\fjoin$, the \emph{join} of two functions, $f$ and $g$ by:

\begin{equation}
\label{def:fjoin}
f \fjoin g =  
  \left\{
     \begin{array}{lr}
       f(x) & \text{if } g(x) = \bot \\
       g(x) & \text{if } f(x) = \bot \\
       \bot & otherwise
     \end{array}
   \right.
\end{equation}
\end{definition}
\todo[inline]{Is there a way to define without using piecewise functions?}

which would allow us to re-write our previous definition of (\ref{eq_fP}) as:

\begin{equation}
f = \restrict{f}{P_1} \fjoin \restrict{f}{P_2} \fjoin ... \fjoin \restrict{f}{P_n}
\end{equation}

\todo[inline]{To Do: problem with this approach + join}

But we must be careful as this definition is not associative.

Let $x \in A \cap B \cap C$, then $( (\restrict{f}{A} \fjoin \restrict{g}{B} ) \fjoin \restrict{h}{C} )(x) = h(x)$ but $( \restrict{f}{A} \fjoin ( \restrict{g}{B} \fjoin \restrict{h}{C} ))(x) = f(x)$

Other conventions exist, for example \emph{Maple}'s 
\texttt{piecewise(cond\_1, f\_1, cond\_2, f\_2, ..., cond\_n, f\_n, f\_otherwise)} 
effectively uses a short-circuted $\fjoin$;
it takes the first sub-function, \texttt{f\_i}, such that the corresponding condition, \texttt{cond\_i}, evaluates to \emph{true}.

This approach simply trades associativity for commutivity.

In this section we will construct a formal system to manipulate partial functions more elegantly.










\section{Hybrid Sets}


\todo[inline]{To Do: join}%%%%%%%%%%%%%%%%%%%%%%%%%%%%%%%%%
We shall consider \emph{hybrid sets}: an extension of multisets which has multiplicities ranging over $\mathbb{N}_0$, a hybrid set has multiplicities over all of $\mathbb{Z}$.

However first we must establish \emph{partial sets} (unrelated to a \emph{poset} or ``partially ordered set'').

Similar to a partial function being only partially defined over its domain, for a partial set, there may exist some items for which membership is undefined.



For an underlying set $U$ we will consider a hybrid set as a function $U \to \mathbb{Z}$ as a way to track the multiplicities of any particular element.

\begin{definition}
Let $U$ be a universe, then any function $U \to \mathbb{Z}$ is called a \emph{hybrid set}.
\end{definition}

On its own, this definition does us very little good; much of the usefulness of sets is derived from their rich notation.

\begin{definition}
Let $H$ be a hybrid set. Then we say that $H(x)$ is the \emph{multiplicity} of the element $x$. We write, $x \in^n H$ if $H(x)=n$. Furthermore we will use $x \in H$ to denote $H(x)\neq 0$ (or equivalently, $x \in^n H$ for $n\neq 0$).
Conversely, $x \notin H$ denotes $x \in^0 H$ or $H(x)=0$.
The symbol $\emptyset$ will be used to denote the empty hybrid set for which all elements have multiplicity 0.
Finally the support of a hybrid set, is the (non-hybrid) set $\text{supp }H$ where $x \in \text{supp }H$ if and only if $x \in H$
\end{definition}



We will use the notation:
\begin{equation*}
H = \hset{x_1^{m_1}, x_2^{m_2},...}
\end{equation*}
to describe the hybrid set $H$ where the element $x_i$ has multiplicity $m_i$. 
We allow for repetitions in $\{ x_i \}$ but interpret the overall multiplicity of an element $x_i$ by the sum of multiplicities among copies. Using Iverson brackets:
\begin{equation}
H(x) = \sum_{x_i \in^{m_i} H} [x = x_i] \; m_i
\end{equation}
For example, $H=\hset{a^1, a^1, b^{-2}, a^3, b^1} = \hset{a^5, b^{-1}}$. 
A writing in which $x_i \neq x_j$ for all $i \neq j$ is refered to as a \emph{normalized form} of a hybrid set. 
For normalized hybrid sets it follows that $H(x_i) = m_i$.



Traditional sets use the operations $\cup$ union, $\cap$ intersection, and $\setminus$ complementation.
In the same way a hybrid set is a function $H : U \to \mathbb{Z}$, a set could be considered as function $S : U \to \{ 0,1 \}$.
Then set operations correspond to pointwise \texttt{OR}, \texttt{AND}, and \texttt{NOT}.
That is, for two sets $A$ and $B$, then $(A \cup B)(x) = A(x) \;\mathtt{OR}\; B(x)$.
One could easily extend union and intersection to hybrid sets using pointwise min and max
%%%%%%%%%%%%%%%%%%%%%%%%%%
[cite],
but it would make more sense to have operations corresponding to primitive operations in $\mathbb{Z}$ instead.
Thus we will define $\oplus$, $\ominus$, and $\otimes$ by pointwise $+$, $-$, and $\cdot$.

\begin{definition}
For any two hybrid sets $A$ and $B$ over a common universe $U$, we define the operations $\oplus, \ominus, \otimes : \mathbb{Z}^U \times \mathbb{Z}^U \to \mathbb{Z}^U$ such that for all $x \in U$:
\begin{equation}
(A \oplus B)(x) = A(x) + B(x)
\end{equation}
\begin{equation}
(A \ominus B)(x) = A(x) - B(x)
\end{equation}
\begin{equation}
(A \oplus B)(x) = A(x) \cdot B(x)
\end{equation}
We also define, $\ominus A$ as $\emptyset \ominus A$ and for $c \in \mathbb{Z}$:
\begin{equation}
(cA)(x) = c \cdot A(x)
\end{equation}
\end{definition}

\begin{definition}
We say $A$ and $B$ are \emph{disjoint} if and only if $A \otimes B = \emptyset$
\end{definition}

Taken alone, hybrid sets can be used to model various objects. 

\subsection{Example: \emph{Rational Arithmetic}}
Any positive rational number can be represented as a hybrid set over the set of primes and vice versa  
(i.e. ($\mathbb{Z}^\mathbb{P}, \oplus) \simeq (\mathbb{Q}_+,\cdot)$).
For any rational number $a/b$, both $a$ and $b$ being integers will have a prime decomposition: $a=p_1^{m_1}\cdot p_2^{m_2} \cdot ...$ and $b=q_1^{n_1} \cdot q_2^{n_2} \cdot ...$ . 
Then there is an isomorphism:
\begin{equation}
f(a/b) = \hset{p_1^{m_1}, p_2^{m_2}, ...} \ominus \hset{q_1^{n_1}, q_2^{n_2}, ...}
\end{equation}

\begin{example}
Concretely, we have:
\begin{equation*}
 20/9 \cdot 15/8 = \hset{5^1, 2^2, 3^{-2}} \oplus \hset{5^1, 3^1, 2^{-3}} = \hset{5^2, 2^{-1}, 3^{-1}} = 25/6
\end{equation*}
\end{example}

%%%%%%% word choice: ``consolidate''
Typically one would need to specify equivalence classes on $\mathbb{Q}$ to consolidate the identity $ca/cb = a/b$.
With hybrid set representation, this identity comes for free:
any common factor between numerator and denominator will cancel result in cancelling multiplicities. 
For example, $2/4 = \hset{2^1, 2^{-2}}$ which is the un-normalized form of $\hset{2^{-1}} = 1/2$.


\todo[inline]{Is there a (nice!) way to extend this for 0 and negative $\mathbb{Q}$ that preserves uniqueness up to normalization? }%%%%%%%%%%%%%%%%%%%


%Monic polynomial
\subsection{Example: \emph{Rational Polynomials}}

Hybrid sets can also be used to represent the roots and asymptotes of a rational polynomial.

\todo[inline]{Blending}

\begin{example} Concretely: 
\begin{equation}
\frac{(x-2)}{(x-1)^2(x+1)} = \hset{ 2^1, 1^{-2}, -1^{-1} }
\end{equation}
\end{example}

\begin{definition}
Generalized partition
\end{definition}

\todo[inline]{To Do}%%%%%%%%%%%%%%%%%%%%%%%%%%%%%%%%%

\begin{definition}
Reducibility, $\mathcal{R}(H) = \mathrm{supp}(H)$
\end{definition}
any set partition is a generalized set partition

a generalized set partition of a reducible partition is a set position iff each generalized partition is reducible 









\section{Hybrid Functions}

Next we consider functions which have hybrid sets as their domain which we will call hybrid functions

\begin{definition}
A hybrid set over $S \times T$ is called a \emph{hybrid (binary) relation}.
\end{definition}

\begin{example}
Algebra of orderings:

For some ordered set $S$ if we define the hyrbid relation $[\succ] = \hset{ (x,y)^1, (y,x)^{-1} : x \succ y }$.
Immediately we have:
\begin{equation}
[<] = \ominus [>]
\end{equation}

If supp$[\succ] = S \times S$ then $[\succ]$ is a \emph{total ordering}

$[\le] = [<] \oplus [=]$

$[\le] = [=] \ominus [>] = [=] \ominus ( [\ge] \ominus [=] ) = 2[=] \ominus [\ge]$

\end{example}

\begin{definition}
Let $H$ be a hybrid relation. For all $x,y,z$ if $(x,y) \in H$ and $(x,z) \in H$ implies $y=z$ then $H$ is said to be a \emph{hybrid function}.
\todo{reword}%%%%%%%%%%%%%%%%%%%%
\end{definition}

Although this tells us what \emph{is} and \emph{is not} a hybrid function, it is not the most useful definition to work with. Generally, we already have a function in mind which we would like to use over a hybrid domain.

\begin{theorem}
Let $H$ be a hybrid set over $U$, $f:B \to S$ be a function where $B \subseteq U$ and $S$ a set. Then
\begin{equation}
f^H := \bigoplus_{x \in B} H(x) \hset{ (x, f(x) )^1 }
\end{equation}
is a hybrid function.
\todo{theorem? reword}%%%%%%%%%%%%%%%%%%%%
\end{theorem}

This definition of a hybrid function should be less thought of as a true function and more as the \emph{graph of a} function.
To return the functional behavior we extend the definition of $\mathcal{R}$ from the previous section.

\begin{definition}
If $H$ is a reducible hybrid set, then $f^H$ is a reducible. Additionally, if $f^H$ is reducible, we override $\mathcal{R}$ by:
\begin{equation}
\mathcal{R}(f^H)(x) = \restrict{f}{\text{supp}(H)}(x)
\end{equation}
\end{definition}

Note that $\mathcal{R}$ can only be applied if at all points $H(x)$ is 0 or 1; 
an irreducible hybrid function cannot be reduced!
Unlike the join for regular functions, $\fjoin$ (\ref{def:fjoin}), the join of hybrid function is simply defined and does not rely on piecewise functions buried in definitions.

\begin{definition}
The \emph{join}, $f^F \hjoin g^G$ of two hybrid functions $f^F$ and $g^G$ is the hybrid relation given by:
\begin{equation}
\label{def:hjoin}
f^F \hjoin g^G := f^F \oplus g^G
\end{equation}
The join of two hybrid relations is identically defined.
\end{definition}

It is important to note that the join operator is closed under hybrid relations but not under hybrid functions.
For any two hybrid functions the result will be a hybrid relation but not necessarily another hybrid function.
So this definition is still nearly as ``dangerous'' as  $\fjoin$, non-hybrid function join. 
We must still be wary of overlapping regions but there are some cases where we can be guaranteed to get a hybrid function.

\todo[inline]{All of the everything} %%%%%%%%%%%%%%%%%%%%%%%%%%%%%%%%%%%

Let $A$ and $B$ be hybrid sets over $U$ and let $f: U \to S$ a function.

$f^A \hjoin f^B = f^{A \oplus B}$ is always a hybrid function. 
Every element in $\text{supp}(f^A \hjoin f^B)$ is using the same map $f$, so there cannot be disagreement among points.

Inductively, this holds for any number of pieces.

For any generalized partition $P$, given by $P = P_1 \oplus P_2 \oplus ... \oplus P_n$, we have
\begin{equation}
 f^P = f^{P_1} \hjoin f^{P_2} \hjoin ... \hjoin f^{P_n}
\end{equation}

For $g:U \to S$ another function then $f^A \hjoin g^B = (f \fjoin g)^{A \oplus B}$ if and only if $A$ and $B$ are disjoint (that is, $A \otimes B = \emptyset$).

But the join of two non-disjoint functions may still be a hybrid function even if their respective functions do not agree at all points; as long as they agree on all points in the ``intersection'' the functions can be safely joined.

\begin{definition}
We say that two hybrid functions \emph{$f^A$ and $g^B$ are compatible} if and only if $f(x) = g(x)$ for all $x \in \text{supp} (A \otimes B)$.
\end{definition}

As with our definition of disjointness, the pointwise product $\otimes$, of hybrid sets acts as an analog for intersection $\cap$, of sets.

Note that any two hybrid functions with disjoint domains will be compatible.


\begin{theorem}
Let $f^A$ and $g^B$ be two hybrid functions. Then $f^A \hjoin g^B$ is a hybrid function if and only if $f^A$ and $g^B$ are compatible.
\end{theorem}

Notion of compatibility is not associative.

Consider

\begin{equation}
(f^H \hjoin g^H) \hjoin g^{\ominus H} = f^H \hjoin (g^H \hjoin g^{\ominus H}) = f^H \hjoin g^\emptyset = f^H
\end{equation}

Although $f^H$ and $g^H$ may be mutually incompatible, their join is compatible with $g^{\ominus H}$.

Similarly, we can also see above that \emph{reducibility} does not lift through $\hjoin$.

\todo[inline]{Is \emph{lift} the right word? - I don't know enough category theory}

\newpage \addtocounter{page}{1}

\subsection{Example: \emph{Piecewise functions on generalized partitions}} 
(2 pages)
\begin{align*}
(f * g) (x) &= \hset{ f_1(x)^{A_1} , f_2(x)^{A_2}} * \hset{g_1(x)^{B_1} , g_2(x)^{B_2}} \\
 &= \hset{(f_1(x)*g_1(x))^{A_1}} \hjoin[*] \hset{(f_2(x)*g_1(x))^{B_1 \ominus A_2}} \hjoin[*] \hset{(f_2(x) * g_2(x))^{B_2}}
\end{align*}

Formula for $f*g$ where $f = f_1^{P_1} \hjoin f_2^{P_2} \hjoin f_n^{P_n}$ and $g = g_1^{Q_1} \hjoin ... \hjoin g_m^{Q_m}$

\begin{align*}
f*g = & \hset{ (f_1 * g_m)^{P_1} } \hjoin[*] ... \hjoin[*] \hset{ (f_{n-1} * g_m)^{P_{n-1}}} \\ \hjoin[*]
& \hset{ (f_n * g_1)^{Q_1} } \hjoin[*] ... \hjoin[*] \hset{ (f_n * g_{m-1})^{Q_{m-1}}} \\ \hjoin[*]
&\hset{ (f_n * g_n)^{U \ominus (P_1 \oplus ... \oplus Q_{n-1} \oplus Q_1 \oplus ... \oplus Q_{m-1})}}
\end{align*}

\todo[inline]{Can't actually do this example until $\hjoin[*]$ is introduced, which doesn't work with basic hybrid functions}



\newpage














\section{Pseudo-functions and Hybrid Forms}
\begin{definition}
Refinement
\end{definition}

\begin{example}
Refinement of intervals
\end{example}

\begin{definition}
Pseudo-function
\end{definition}

Properties of pseudo-functions

\begin{definition}
Hybrid Form
\end{definition}

\begin{definition}
$(f^F \hjoin[*] g^G)(x) = (F(x) + G(x)) \hset{(x, (f*g)(x))^1}$
\end{definition}

Returning to the example of the sign function from section 2.1, we could think of the function as the join of 3 different hybrid functions:
\begin{equation}
\text{sign} = -1^{(-\infty, 0)} \hjoin 0^{\{0\}} \hjoin 1^{(0, \infty)}
\end{equation}

but we could also consider it as the ``joined sum'' of $-1^{(-\infty, 0]}$ and $1^{[0, \infty)}$.

(3 pages)

\newpage

\subsection{Example: \emph{Piecewise functions revisited}}
Repeat piece-wise function example with unsafe points (1 page)
\begin{equation}
(-x^2+2)^{[-1,1]} \hjoin \left( \frac{1}{x^2} \right)^{\mathbb{R} \ominus [1,1]}
\end{equation}


\newpage