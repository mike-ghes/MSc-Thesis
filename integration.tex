\def\ind[#1]{\mathbb{I}_{#1}}
\def\extendedreal{\bar{\mathbb{R}}}
\def\hsetover[#1]{\mathbb{Z}^{#1}}

\chapter{Integration III} \label{integration}
\doublespacing

``Chapter goals''

Section 4.1 will cover a conventional treatment of the Lebesgue integral, for those already familiar with the construction, this may be skipped. 

%%%%%%%%%%%%%%%%%%%%%%%%%%%%%%%%%%%%%%%%%
%
% LEBESGUE INTEGRATION
%
%%%%%%%%%%%%%%%%%%%%%%%%%%%%%%%%%%%%%%%%%

\section{Sigma Algebras}

Before we can talk about the Lebesgue integral we must first set the stage, so to speak.



\begin{definition}
Let $X$ be a non-empty set. A \textbf{$\sigma$-algebra on the set $X$}, $\Sigma$,  is a family of subsets of $X$ such that:
\begin{enumerate}
\item $\Sigma$ is non-empty
\item \emph{Closed under complement.} If $E \in \Sigma$, then $X \setminus E \in \Sigma$.
\item \emph{Closed under countable union.} If $E_1, E_2, ... \in \Sigma$ then $(E_1 \cup E_2 \cup ... ) \in \Sigma$.
\end{enumerate}
The pair $(X, \Sigma)$ is called a \textbf{measurable space} and elements of $\Sigma$ are called the \textbf{measurable sets} (of $X$).
\end{definition}

It can easily be shown through the use of De Morgan's laws that a $\sigma$-algebra is also closed under countable intersection as well.
\begin{example}
$\{ \emptyset, X \}$ is a $\sigma$-algebra on $X$. In fact, $X$ and $\emptyset$ are members of \emph{every} $\sigma$-algebra on $X$.
\end{example}

\begin{example}
$2^X$ is a $\sigma$-algebra on $X$.
\end{example}

However, we would also like to be able to construct more interesting $\sigma$-algebras.

\begin{definition}
Given an arbitrary family of subsets $F \subseteq 2^X$, there is a unique smallest $\sigma$-algebra containing $F$ which is called the \textbf{$\mathbf{\sigma}$-algebra generated by $F$} and we will denote as $\sigma(F)$.
\end{definition}

Can be constructed by taking the intersection of all $\sigma$-algebras containing $F$.

Of particular interest to us is the $\sigma$-algebra generated by a topology, $\mathcal{T}(X)$. 

Borel $\sigma$-algebra: $\mathcal{B}(X) = \sigma(\mathcal{T}(X))$


