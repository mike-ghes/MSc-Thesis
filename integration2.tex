\chapter{Integration II}

Earlier we established integration of functions and treated the ``$dx$'' nearly as punctuation.
Another more flexible and approach is to use \emph{differential forms}.




%%%%%%%%%%%%%%%%%%%%%%%%%%%%%%%%%%%%%%%%%
%
% DIFFERENTIAL FORMS
%
%%%%%%%%%%%%%%%%%%%%%%%%%%%%%%%%%%%%%%%%%

\section{Differential Forms}

\begin{definition}
	A \textbf{(differential) 0-form} $\beta$ on $\mathbb{R}^n$ is a function $\beta : \mathbb{R}^n \to \mathbb{R}$.
	A \textbf{(differental) 1-form} $\omega$ on $\mathbb{R}^n$ is an expression of the form:
	\begin{equation}
		\omega = f_1(\text{x}) \; dx_1 + f_2(\text{x}) \; dx_2 + \ldots + f_n(\text{x}) \; dx_n
	\end{equation}
\end{definition}

There is very little to say about 0-forms, they are just functions on $\mathbb{R}^n$.
1-forms look very much like something we're used to integrating with.
Of course if all but one of $f_i$ are zero then we have a \emph{basic 1-form}, $\omega = f_i \; dx_i$ which is nothing new to integrate over.
Using the linearity of integration, general 1-forms can easily be seperated into a sum of integrals on basic 1-forms.
Typically, 1-forms are encountered inexplicitly in a multivariate calcululus class with Green's theorem:
\begin{equation}
	\tag{Green's Theorem}
	\int_D \left( \frac{\partial f_2}{\partial x} - \frac{\partial f_1}{\partial y}  \right) \;dx\;dy 
	=\int_{\partial D} f_1(x,y) \; dx + f_2(x,y) \; dy
\end{equation}

\todo[inline]{
Formally, $T_x(\mathbb{R}^n) \to \mathbb{R}$.


}


We can add differential forms but rather than multiplication we use the $\wedge$ product.

To create higher order $p$-forms we use the wedge operator $\wedge$. 

First of all, the wedge product is primarily defined by being \emph{anti-commutative} or \emph{skew-symmetric}.
That is, $dx \wedge dy = -dy \wedge dx$.
Several results immediately follow from this.
When used on two identical $dx$, we have $dx \wedge dx = - dx \wedge dx$ and so the result must be zero.
Additionally, for any permutation $\sigma$ of $[p]$:
\begin{equation}
	dx_1 \wedge ... \wedge dx_p = \text{sgn}(\sigma) \; dx_{\sigma(1)} \wedge ... \wedge dx_{\sigma(p)}
\end{equation}


\begin{definition}
	Given a $k$-rectangle $\Omega \in \mathbb{R}^n$ with coordinates $\text{x} = (x_1, x_2, \ldots, x_n)$
	A \textbf{differential $p$-form} $\beta$ over $\Omega$ has the form:
	\begin{equation}
		\beta = \sum_{j_1 \in [n]} \ldots \sum_{j_p \in [n]} b_{(j_1, \ldots, j_p)}(\text{x}) \; 
				\text{d} x_{j_1} \wedge \ldots \wedge \text{d} x_{j_p}
	\end{equation}
	Typically, we will take $j = (j_1, \ldots, j_p)$ and express $\beta$ as, 
	$\sum_j b_j(x) \; dx_{j_1} \wedge \ldots \wedge dx_{j_p}$.
	We denote the space of all $p$-forms on $\Omega$ by $\Lambda^p(\Omega)$.
\end{definition}


$dx^i$ and $dx^j$ are trivial differential 1-forms by taking $f_i=1$ and $f_j=1$ respectively.
When we take the wedge product of these, we end up with the 2-form $dx^i \wedge dx^j$.
It should then seem natural to think of $\wedge$ as an operator which maps 
a $k$-form and an $\ell$-form to a $(k+\ell)$-form.
For general $p$ and $q$-forms we define the wedge product.

\begin{definition}
	Let $\alpha = \sum_i a_i(x) \; dx_{i_1} \wedge \ldots \wedge dx_{i_p} \in \Lambda^p(\Omega)$ and 
	$\beta = \sum_j b_j(x) \; dx_{j_1} \wedge \ldots \wedge dx_{j_q} \in \Lambda^q(\Omega)$. We extend the
	wedge product $\wedge : \Lambda^p(\Omega) \times \Lambda^q(\Omega) \to \Lambda^{p+q}(\Omega)$ by:
	\begin{equation}
		\alpha \wedge \beta  = \sum_{i,j} a_i(x) b_j(x) \; 
			dx_{i_1} \wedge \ldots \wedge dx_{i_p} \wedge 
			dx_{j_1} \wedge \ldots \wedge dx_{j_q}
	\end{equation}
\end{definition}

Most of the possible terms will end up being zero.
For any $q+p > n$, no term will be contributed.
If any of $dx_{i} = dx_{j}$ then no term will be contributed either.

The wedge operator is not skew symmetric with higher order differential-forms.
Rather we have:
\begin{equation}
	\alpha \wedge \beta = (-1)^{pq} \beta \wedge \alpha
\end{equation}
This can be easily seen by commuting each of the $q$, $dx_{j}$ terms each past the $p$, $dx_{i}$ terms.


Several other nice identities:

If $f$ is a function, $\omega_1$ and $\omega_2$ are $k$-forms, $\eta$ an $m$-form and $\tau$ a form then.
\begin{align}
	(\omega_1 + \omega_2) \wedge \eta  & \;=\; \omega_1 \wedge \eta + \omega_2 \wedge \eta \\
	(\omega_1 \wedge \eta) \wedge \tau & \;=\; \omega_1 \wedge ( \eta \wedge \tau ) \\
	(f \cdot \omega_1) \wedge \eta & \;=\;  f \cdot (\omega_1 \wedge \eta) \;=\; \omega_1 \wedge (f \cdot \eta)
\end{align}


These should all be quite obvious from definitions. 

Integrating over a $p$-form is quite simple.

Generally we assume a $p$-form and a $p$-rectangle.

Then we simply treat $dx_1 \wedge dx_2 \wedge \ldots \wedge dx_n$ as $dx_1 \;dx_2 \ldots dx_n$. 

\todo[inline]{what about $\int .. dx \wedge dy = \int .. dxdy = -\int .. dy \wedge dx = -\int dydx$}

\begin{definition}
Let $\alpha$ be a $k$-form on $\Omega \subset \mathbb{R}^n$ of the form $\alpha = A(x) \; \text{d}x_1 \wedge ... \wedge \text{d} x_n$.
If $A \in \mathcal{L}^1 (\Omega , \text{d}x)$ then we define:
\begin{equation}
\int_\Omega \alpha = \int_\Omega A(x) \; \text{d}x
\end{equation}
Where the left-hand side is the integral of a $k$-form and the right-hand side is a Lebesgue integral.
For any $\beta \in \Lambda^k (\Omega)$ we extend this definition linearly as the sum of integrals.
\end{definition}



%%%%%%%%%%%%%%%%%%%%%%%%%%%%%%%%%%%%%%%%%
%
% PULL-BACKS (or coordinate changes)
%
%%%%%%%%%%%%%%%%%%%%%%%%%%%%%%%%%%%%%%%%%

\section{Pull-backs}

Benefit of differential forms is how cleanly they handle changes in coordinates.

This is generally done through the use of pull-backs.

A pullback $F^* \omega$ can be thought of as $\omega(F(x))$.

Gets its name from pulling $F$ back through $\omega$

\begin{definition}
$F: X \to \Omega$
Define the pullback $F^* \beta$
\begin{equation}
F^* \beta = \sum_j  b_j ( F(x)) (F^* \text{d}x_{j_1}) \wedge ... \wedge (F^* \text{d} x_{j_k})
\end{equation}
and
\begin{equation}
F^* \text{d}x_j = \sum_\ell \frac{\partial F^j}{\partial x_\ell} \; \text{d} x_\ell
\end{equation}
\end{definition}

Which can be reduced by:
\begin{align}
F^ * \beta & = \sum_j  b_j ( F(x)) (F^* \text{d}x_{j_1}) \wedge ... \wedge (F^* \text{d} x_{j_k}) \\
& = \sum_j  b_j ( F(x))  
\left( \sum_\ell \frac{\partial F^{j_1}}{\partial x_\ell} \; \text{d} x_\ell \right)
\wedge ... \wedge  
\left( \sum_\ell \frac{\partial F^{j_k}}{\partial x_\ell} \; \text{d} x_\ell \right) \\
& = ... \\
& = \sum_j b_j ( F(x)) \; \text{det}\left( J_F \right) \; \text{d}x_{j_1} \wedge ... \wedge \text{d} x_{j_k}
\end{align}

Which is significant given the change of variable formula for integration:

\begin{equation}
\int_{\phi(U)} \! f(v) \; dv = \int_U \! f(\phi(u)) \; |\text{det}\phi'(u)| \; du
\end{equation}

\begin{theorem}
Let $F : X  \to \Omega$ be an (orientation-preserving diffeomorphism) and $\alpha$ an integrable $n$-form on $\Omega$ then
\begin{equation}
\int_\Omega \alpha = \int_X F^* \alpha
\end{equation}
\end{theorem}

More algebra of differential forms

\begin{equation}
F^* (\alpha \wedge \beta ) = (F^* \alpha) \wedge (F^* \beta)
\end{equation}

\begin{definition}
Exterior derivative
\end{definition}

...

\begin{equation}
d(\alpha \wedge \beta) = (d \alpha) \wedge \beta + (-1)^j \alpha \wedge ( d \beta)
\end{equation}

...

\begin{equation}
F^* (d \beta ) = dF^* \beta
\end{equation}


\todo[inline]{Unapologetic Mathematician - pullbacks, manifold integration...}

%%%%%%%%%%%%%%%%%%%%%%%%%%%%%%%%%%%%%%%%%
%
% MANIFOLD INTEGRATION
%
%%%%%%%%%%%%%%%%%%%%%%%%%%%%%%%%%%%%%%%%%
\section{Integration over Manifolds}

A manifold is nothing more than a collection of local charts diffeomorphic to $\mathbb{R}^n$

So we use pullbacks to turn a set of rectangles into a manifold while simultaneously applying the manifold's pullback.

Integration on oriented manifolds is just a careful application of Theorem 5.2.1

\begin{example}
Integrate an atlas with overlapping charts (using inclusion-exclusion)
\end{example}



%%%%%%%%%%%%%%%%%%%%%%%%%%%%%%%%%%%%%%%%%
%
% STOKE'S THEOREM
%
%%%%%%%%%%%%%%%%%%%%%%%%%%%%%%%%%%%%%%%%%
\section{Stokes' Theorem}


%%%%%%%%%%%%%%%%%%%%%%%%%%%%%%%%%%%%%%%%%
% Contour Integral
%%%%%%%%%%%%%%%%%%%%%%%%%%%%%%%%%%%%%%%%%
\subsection{Example: \emph{Contour Integration}}
\todo[inline]{??????????}

Using Stokes' theorem and Inclusion/Exclusion to evaluate a tricky theorem like:

\begin{equation}
f(z) = \frac{z^2}{(z^2 + 2z + 2)}
\end{equation}

% http://en.wikipedia.org/wiki/Methods_of_contour_integration

 (2-3 pages)

\newpage
