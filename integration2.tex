\chapter{Integration II}


%%%%%%%%%%%%%%%%%%%%%%%%%%%%%%%%%%%%%%%%%
%
% DIFFERENTIAL FORMS
%
%%%%%%%%%%%%%%%%%%%%%%%%%%%%%%%%%%%%%%%%%

\section{Differential Forms}

\begin{definition}
A $k$-form $\beta$ on the open set $\Omega \subset \mathbb{R}^n$ has the form:
\todo[inline]{open set or Lebesgue measurable sets??}
\begin{equation}
\beta = \sum_j b_j(x) \; \text{d} x_{j_1} \wedge ... \wedge \text{d} x_{j_k}
\end{equation}
where $j=(j_1, ..., j_k)$ is a $k$ dimensional multi-index. We say that $\beta \in \Lambda^k(\Omega)$
\end{definition}

We have not yet defined the $\wedge$ operator.

Anti-commutative: $\text{d} x \wedge \text{d} y = - \text{d}y \wedge \text{d} x$. 
Which implies for any permutation $\sigma$ of $\{1,...,k\}$:
\begin{equation}
dx_1 \wedge ... \wedge dx_k = \text{sgn}(\sigma) \; dx_{\sigma(1)} \wedge ... \wedge dx_{\sigma(k)}
\end{equation}

Anti-commutativity additionally implies that for all $x_i$, $dx_i \wedge dx_i = 0$. 

Let $\alpha = \sum_i a_i(x) \; dx_{i_1} \wedge ... \wedge dx_{i_\ell} \in \Lambda^\ell(\Omega)$ and
$\beta = \sum_j b_j(x) \; \text{d} x_{j_1} \wedge ... \wedge \text{d} x_{j_k} \in \Lambda^k(\Omega)$ then define:

\begin{equation}
\alpha \wedge \beta  := \sum_{i,j} a_i(x) b_j(x) \; dx_{i_1} \wedge ... \wedge dx_{i_\ell} \wedge dx_{j_1} \wedge ... \wedge dx_{j_k}
\end{equation}

Thus we can think of $\wedge$ as mapping a $k$-form and an $\ell$-form to a $(k+\ell)$-form, $\wedge : \Lambda^\ell(\Omega) \times \Lambda^k (\Omega) \to \Lambda^{k+\ell} (\Omega)$. By anti-commutativity we have:

\begin{equation}
\alpha \wedge \beta = (-1)^{k \ell} \beta \wedge \alpha
\end{equation}


\begin{definition}
Let $\alpha$ be a $k$-form on $\Omega \subset \mathbb{R}^n$ of the form $\alpha = A(x) \; \text{d}x_1 \wedge ... \wedge \text{d} x_n$.
If $A \in \mathcal{L}^1 (\Omega , \text{d}x)$ then we define:
\begin{equation}
\int_\Omega \alpha = \int_\Omega A(x) \; \text{d}x
\end{equation}
Where the left-hand side is the integral of a $k$-form and the right-hand side is a Lebesgue integral.
For any $\beta \in \Lambda^k (\Omega)$ we extend this definition linearly as the sum of integrals.
\end{definition}
\todo[inline]{Define $dx$ from $x_1 , ... , x_n$. Need to lift sign change from permutations}


\newpage
%%%%%%%%%%%%%%%%%%%%%%%%%%%%%%%%%%%%%%%%%
%
% PULL-BACKS (or coordinate changes)
%
%%%%%%%%%%%%%%%%%%%%%%%%%%%%%%%%%%%%%%%%%

\section{Pull-backs}

Benefit of differential forms is how cleanly they handle changes in coordinates.

\begin{definition}
$F: X \to \Omega$
Define the pullback $F^* \beta$
\begin{equation}
F^* \beta = \sum_j  b_j ( F(x)) (F^* \text{d}x_{j_1}) \wedge ... \wedge (F^* \text{d} x_{j_k})
\end{equation}
and
\begin{equation}
F^* \text{d}x_j = \sum_\ell \frac{\partial F^j}{\partial x_\ell} \; \text{d} x_\ell
\end{equation}
\end{definition}

Which can be reduced by:
\begin{align}
F^ * \beta & = \sum_j  b_j ( F(x)) (F^* \text{d}x_{j_1}) \wedge ... \wedge (F^* \text{d} x_{j_k}) \\
& = \sum_j  b_j ( F(x))  
\left( \sum_\ell \frac{\partial F^{j_1}}{\partial x_\ell} \; \text{d} x_\ell \right)
\wedge ... \wedge  
\left( \sum_\ell \frac{\partial F^{j_k}}{\partial x_\ell} \; \text{d} x_\ell \right) \\
& = ... \\
& = \sum_j b_j ( F(x)) \; \text{det}\left( J_F \right) \; \text{d}x_{j_1} \wedge ... \wedge \text{d} x_{j_k}
\end{align}

Which is significant given the change of variable formula for integration:

\begin{equation}
\int_{\phi(U)} \! f(v) \; dv = \int_U \! f(\phi(u)) \; |\text{det}\phi'(u)| \; du
\end{equation}

\begin{theorem}
Let $F : X  \to \Omega$ be an (orientation-preserving diffeomorphism) and $\alpha$ an integrable $n$-form on $\Omega$ then
\begin{equation}
\int_\Omega \alpha = \int_X F^* \alpha
\end{equation}
\end{theorem}

More algebra of differential forms

\begin{equation}
F^* (\alpha \wedge \beta ) = (F^* \alpha) \wedge (F^* \beta)
\end{equation}

\begin{definition}
Exterior derivative
\end{definition}

...

\begin{equation}
d(\alpha \wedge \beta) = (d \alpha) \wedge \beta + (-1)^j \alpha \wedge ( d \beta)
\end{equation}

...

\begin{equation}
F^* (d \beta ) = dF^* \beta
\end{equation}

\section{Integration over Manifolds}

\begin{example}
Integrate an atlas with overlapping charts (using inclusion-exclusion)
\end{example}
\newpage

\section{Stokes' Theorem}



\subsection{Example: \emph{Contour Integration}}
\todo[inline]{??????????}

Using Stokes' theorem and Inclusion/Exclusion to evaluate a tricky theorem like:

\begin{equation}
f(z) = \frac{z^2}{(z^2 + 2z + 2)}
\end{equation}

% http://en.wikipedia.org/wiki/Methods_of_contour_integration

 (2-3 pages)

\newpage
